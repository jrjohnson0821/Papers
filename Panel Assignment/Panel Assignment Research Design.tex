\documentclass[12pt]{article}
\usepackage{setspace}
\usepackage{mathtools}
\usepackage{nicefrac}
\usepackage{fullpage}
\usepackage{times}
\usepackage{mathptmx}
\usepackage{graphicx}
\usepackage{natbib}
\usepackage{listings}
\usepackage{float}
\usepackage{wrapfig}
\usepackage{lscape}
\usepackage{enumitem}
\usepackage{hyperref}
\usepackage[affil-it]{authblk}
\usepackage{color}
\usepackage{todonotes}

%setup hyperlinks color
\definecolor{PSU}{RGB}{0,0,153}
\hypersetup{
	colorlinks=true,       % false: boxed links; true: colored links
	linkcolor=black,          % color of internal links (change box color with linkbordercolor)
	citecolor=PSU,        % color of links to bibliography
	filecolor=PSU,      % color of file links
	urlcolor=PSU           % color of external links
}

\title{Reassessing Random Assignment on the U.S. Courts of Appeals\footnote{This paper is part of a larger project for which a grant was received from The Pennsylvania State University Department of Political Science.  All errors and omissions are the responsibility of the author.} }

\author{\href{http://www.jeremyrjohnson.org/}{Jeremy R.\ Johnson}\\ \href{mailto:Jeremy.Johnson@psu.edu}{Jeremy.Johnson@psu.edu}}
\affil{Pennsylvania State University}
\date{\today}




\begin{document}
\clearpage\maketitle\thispagestyle{empty}


%\begin{abstract}
%	\medskip
%	This is a test of the emergency abstract placeholder system.  If this had been an actual abstract, it would have briefly summarized the main points of the paper, so that a casual reader could quickly understand what the paper is about.  A high-quality abstract may also entice that person to read the entire paper, and possibly even to think about its contents.  This has been a test of the emergency abstract placeholder system. \\
%\end{abstract}

\clearpage
\setcounter{page}{1}
\setcounter{footnote}{0}
\renewcommand*{\thefootnote}{\arabic{footnote}}
\addtolength{\footnotesep}{6pt}


\section{Introduction}
It has been long argued that federal judges are not unbiased, but are swayed by their political and ideological leanings \citep{segal2002supreme}.  However, the United States federal judiciary is considered to be one of the most independent of all countries \citep{Linzer2014}.  One of the hallmarks of that system is that judges in both the trial and appellate level are randomly assigned.  This prevents both litigants and judges from improperly swaying a judicial proceeding.  This random assignment of judges is assumed to be standard operating procedure.  However, recently studies have disputed this finding, showing that the observed partisan makeup of panels was not very likely statistically \citep{Chilton2014}.  We will expand on the finding by \citeauthor{Chilton2014} and seek to find a more robust explanation for panel assignment, looking at individual judge characteristics rather than simply looking at partisanship as did \citet{Chilton2014}.

The United States Circuit Courts of Appeal (COA) are some of the most important federal courts in the United States.  They are second only to the United States Supreme Court in their influence, and some would argue even more so.  The Supreme Court hears less than 150 cases a year, while the COA hear over 50,000 per year \citep{Caseloadstats}.  Precedents set by these courts are biding over wide geographic areas, and due to the low case acceptance rate by the Supreme Court are often the last word on federal cases.

There are numerous consequences of finding that COA panels and cases are not randomly assigned.  The primary consequence is that a following an assignment process that differs from the publicly declared process for assigning appellate panels violates the normative theory of an independent, transparent judiciary.  The stated position of the federal judiciary is that panels and cases are randomly assigned \citep{Journalist2011}.\footnote{This is also stated in various forms throughout the Rules and Procedures of the Circuits.}  A second normative consequence is that if panels are not randomly assigned, why doesn't the judiciary simply admit that is the case and publish the procedures for assigning appellate panels?  

The third consequence of this lack of transparency is that if there is a non-random method of assigning panels, what is the purpose and biases related to this method?  One of the hallmarks of a democracy is that those in power are accountable to the citizens.  Typically judicial branches are considered a special case in which independence as opposed to accountability is desired.  However, this is premised on the fact that the judges are not corrupt.  If these judges are responding to some other influence, whether it is a desire to influence policy outcomes, or if they are receiving illegal compensations, whether in the form of promotions or payments, this would seriously bring in to question the equitably of justice. 

\section{Previous Studies}\label{Previous}
\subsection*{\cite{Atkins1974}}
\citet{Atkins1974} begin with the premise that all COA panels are randomly assigned by the Chief Judge.  \citet{Atkins1974} use a probability assessment to determine whether there is any discrepancy between how panels could have been assigned and how the panels were actually assigned for the Fifth Circuit during the 1960's.\footnote{During this period the Fifth Circuit had jurisdiction over Florida, Georgia, Alabama, Mississippi, Louisiana, and Texas.  These were all states which had a history of discrimination both politically and in the courts towards minorities.}  \citeauthor{Atkins1974} specifically focus on race relations cases.  This was a period in which state judges as well as federal district judges were consistently conservative, especially in race relations cases \citep{Atkins1974}.  Four of the nine judges on this court, most importantly including the Chief Judge Elbert Tuttle were consistently liberal on civil rights issues in contrast to the state, and federal district judges under their jurisdiction.  Tuttle, Wisdom, and Brown were appointed by Eisenhower, while Rives was appointed by Truman \citep{FJC}.   Some have argued that Eisenhower consciously chose racial reformers for key judicial appointments, while others have argued that he was not good at predicting behavior  \citep{kahn1992shattering,ambrose1984eisenhower}.  \cite{Atkins1974} conclude that the liberal minority of the court heard a disproportionate number of cases involving African-American litigants and race relations cases.  Specifically, that they responded favorably to claims raised by black litigants post \textit{Brown v. Board of Education} \citeyearpar{warren1954brown}.

\subsection*{\cite{Brown2000}}
\citet{Brown2000} examine ``panel-packing'' in the Civil Rights Era.  \citeauthor{Brown2000} provides a detailed case study on the Fifth Circuit and the events surrounding the race cases and the alleged panel packing that occurred under the leadership of Chief Judge Tuttle.  \citeauthor{Brown2000} examine the circumstances in which these allegations were made public, both through a dissent in \citet{Armstrong}, and a \textit{Houston Chronicle} newspaper article.  \citeauthor{Brown2000} make clear that during the era of the panel-packing debate, the Chief Judge exercised considerable authority over the composition of panels.  Also during this era, the Clerk of the Court was aware of the panels in which he was assigning cases to \citep{Brown2000}. 

\subsection*{\cite{Chilton2014}}
The most recent and revealing previous study into how judges are assigned cases in the COA is \cite{Chilton2014}.  \citeauthor{Chilton2014} examine all cases for September 2008 through August 2013.  There are a total of 10,364 cases in this dataset.  Similar to the Songer dataset, they examine cases as they actually happened, rather than cases as they were scheduled.  However, one drawback to \citeauthor{Chilton2014}'s work is that they only test for one hypotheses and do not collect relevant data to test for others.  While this is normal for many studies, we believe that this issue is much more complex and requires a more in depth examination of the hypotheses.  While the data amassed in the process of this project was extensive, it collected a great number of observations on a very small number of variables.  \citeauthor{Chilton2014} only examine the partisan make-up of panels to test for random assignment of the partisan make up.  We propose to make up for that deficit of information by not only examining a much longer period of time, but also on many more variables. These will be explained in depth in Sections \ref{Hypotheses} and \ref{Data}.  

\citeauthor{Chilton2014}'s primary question is ``whether the ideological balance of panels is consistent with the balance that would haven be produced by a truly random process,'' \citep[20]{Chilton2014}.  While the partisan balance of panel assignment is obviously important, there there should be much finer tuned hypotheses to examine the effects of the judges such as the Chief Judge and the individual judges.  \citeauthor{Chilton2014} find statistical evidence for non-randomness is four of the twelve circuits, the D.C. Circuit, Second Circuit, Eighth Circuit, and the Ninth Circuit.  The D.C. Circuit has long been known as a stepping stone to the Supreme Court, with four of the current nine justices on the Supreme Court having been promoted from D.C. Circuit \citep{FJC}.  The Ninth Circuit has long been accused of a pronounced liberal bias as well \citep{farris1997ninth,scott2006supreme,herald1998reversed,chemerinsky2003myth,chemerinsky2003myth,goldman1968conflict}.  \citeauthor{Chilton2014} did not find any significant partisan bias in the other circuits.

\section{Theory}\label{Theory}
The United States Courts of Appeals are broken into eleven geographic circuits plus the D.C. Circuit and the Federal Circuit. These courts' primary responsibility is to hear appeals from the 94 U.S. District Courts. Traditionally these courts hear cases in panels of three judges. In normal practice, panels are randomly chosen then a series of cases are randomly assigned to these panels \citep{Hooper2011,Journalist2011,Chilton2014,Songer2007}.  These assignments are made to all current circuit judges and available senior judges.\footnote{Based on need, there are also exceptions when district judges will sit on panels, as well as current and retired Supreme Court justices.} There is variation from circuit to circuit on panel length and the number of cases that each panel hears before judges are assigned to a different panel as well as the number of judges on the circuit; however, the randomness is nominally standard across all circuit courts.  While this process is generally taken to be the normal procedure in the Federal Appellate courts, a substantial literature has challenged this as fact \citep{Atkins1974,Brown2000,Chilton2014}.  

\subsection*{Chief Judges}
The Chief Judge of a Court of Appeals is a very important player in both the politics and administration of a Circuit Court of Appeals.  The Office of Chief Judge of the Circuit was created in 1948 to handle administrative duties for the Circuit \citep{feinberg1984office}.  The Chief Judge is not selected politically, but by seniority.\footnote{According to 28 U.S. Code \S45: The chief judge of the circuit shall be the circuit judge in regular active service who is senior in commission of those judges who:
	\begin{enumerate}[label=(\Alph*)]
	\item are sixty-four years of age or under;
	\item have served for one year or more as a circuit judge; and
	\item have not served previously as chief judge.
\end{enumerate} \citep{Journalist2011}.}  Since this is not a political position, the current president does not control the appointment of the Chief Judge.  Also, since this is based on seniority, the Chief Judge was almost assuredly appointed to the Circuit by a previous President.  This often results in divergent ideology between the Chief Judge and the current President and the Congress.

\citet{feinberg1984office} is a qualitative description of the Office of Chief Judge.  \citet{feinberg1984office} discusses how he assigned panels and cases during his time as the Chief Judge of the Second Circuit.  This description is vastly different that that expounded by the official statements of the federal judiciary.  \citet[374]{feinberg1984office}  discusses how he assigns panels to spread out the duties of presiding over panel, which is assigned to the most senior judge on the panel.  As \citeauthor{feinberg1984office} points out, there are various perks and responsibilities associated with being the presiding judge on a panel.  In addition to assigning the opinion to be written, the presiding judge writes the bulk of the summary orders \citep[375]{feinberg1984office}.  This allows the presiding judge to wield a large amount of influence over the cases that his or her panel is hearing that week.  As \citet{landes1998judicial} point out, a more senior judge can have influence on the outcome of a panel decision.  If this judge was specifically chosen to hear certain cases by the Chief Judge, he or she could have a much larger sway on those specific case outcomes.

\citet[387]{feinberg1984office} describes how efficiency could be achieved by using a computer to assign panels, but how he continues to work with experienced members of the clerk's office to assign panels.  While this is a dated article, written in 1984, it describes a departure from the normative theory that panels should be randomly assigned.

\subsection*{Clerks of the Court}
The Clerk of the Court is a member of the Circuit Executive's Office.  The Circuit Executive is an administrative staff member, who is appointed by the Circuit Judicial Council.\footnote{A Circuit Judicial Council is a governing body in each federal circuit created by Congress to ensure the effective and expeditious administration of justice in that circuit. Each council has an equal number of circuit and district court judges; the chief judge of the circuit is the presiding officer \cite{FJC}.}  As a result of this, we hypothesize that the hiring of the Circuit Executive is often a political or patronage decision of the Chief Judge of the circuit, who arguably has a great deal of sway over this decision.\footnote{As a further extension of this hypothesis, we will examine the turnover rate, as well as hopefully identifying the partisan affiliation of the Circuit Executives and the Clerks of the Circuits.}  As noted by \cite{Atkins1974} the clerk of the court responds directly to orders given by the Chief Judge.  The Clerk of the Court is responsible for assigning cases to panels when they are ready to be heard \cite{Atkins1974,feinberg1984office,FJC}.  This gives the Clerk tremendous power in determining the outcome of cases based on the makeup of the panels that he or she is assigning cases to.  

\section{Hypotheses}\label{Hypotheses}
\subsection*{Ideological Imbalance Hypothesis}
We will build upon \citeauthor{Chilton2014}'s findings, that the ideological balance of panels is not consistent with a truly random process \citep[20]{Chilton2014}.  This finding was not directional in that it did not specify how the ideology of these panels should change based on other characteristics.  We will begin by testing this basic hypothesis, in a similar manner to \citet{Chilton2014}.  We will then expand this finding to account for other factors, as well as test subsequent hypotheses relating to these characteristics.  We expect to find results similar to \citet{Chilton2014} in that \textbf{there will an ideological imbalance in the composition of panels}.

\subsection*{Seniority Hypothesis}
In addition to having more experience on the court in general, more senior judges also have longer periods of time working with the Chief Judge and therefore have a better chance to influence their choices in panel assignment.  Many scholars have noted that appellate courts operate on a strong tradition of collegiality, and therefore we believe that collegiality will allow senior judges to use their relationships with their colleagues, specifically the Chief Judge, to influence their selection to panels \cite{Atkins1974,feinberg1984office,Brown2000}. Senior judges should also tend to sway the junior judges on panels with them toward their ideological position, as noted by \citet{landes1998judicial}.  We therefore hypothesize that \textbf{more senior judges will be in the majority on salient cases} while junior judges will be in the minority on highly salient cases.

\subsection*{Chief Judge Distance Hypothesis}
Despite the theoretically randomness of panel assignment, the Chief Judge of the Circuit has the ability to assign judges who are sitting by designation.\footnote{This could be judges who are District Judges who have been ``called up'' to sit on one case or one panel in the case of scheduling or case-specific recusals of a particular judge.}  As \citet{Atkins1974} note, through this mechanism, as well as the wide array of other administrative responsibilities, the Chief Judge wields a large amount of power in the Circuit.  Therefore, we hypothesize that being a co-partisan with the Chief Judge will enhance a judge's chance of being in the majority on a highly salient case.  Or the reverse of that situation, that a member of the opposite party will make them less likely to be in the minority on a salient case.   

\subsection*{Gender Hypothesis}
On circuits with a female chief judge, other female judges will serve more often on highly salient cases than on circuits with male chief judges.  This effect will have a larger effect when those judges are Democrats.

\subsection*{Racial Hypothesis}
On circuits with African-American or Hispanic chief judges, other minority judges will serve more often proportionally than they should in the majority in highly salient cases, as well as cases which involve civil rights or litigants who are of the same race.  This comes directly from the results found by \citet{Atkins1974} and \citet{Brown2000}.

\subsection*{Clerk Effect Hypothesis}
Clerks of the Court are important to the assigning of panels in a strictly administrative sense.  It will be important to establish how much authority the Clerk has to assign panels as well as cases to panels.  We hypothesize that the appointment process of the Circuit Executive and the Clerk of the Circuit will have a direct effect on the way that panels are assigned.  If the Clerk is appointed by the Chief Judge when he assumes office, in the manner of a patronage appointment, we expect that we will see an effect on how panels are appointed when there is a change in the Circuit Executive or the Clerk of the Court.

\section{Data}\label{Data}
The data for this project will come from the Songer Court of Appeals Database for the case data \citep{Songer2007}.\footnote{All data and documentation for this dataset can be found at \url{http://artsandsciences.sc.edu/poli/juri/appct.htm}}  This dataset is uniquely suited to examine the effects of random assignment of judges.  While the stratification of this data presents an interesting methodological challenge, it cannot be overcome.  This dataset takes all Court of Appeals cases decided since 1925 and randomly selects either 15 or 30 cases to fully code \citep{hurwitz2006institutional,hurtwitz2012changes}.\footnote{This data consists of equal numbers of cases either 15 or 30 from each circuit-year.  The sample is equally drawn from each circuit-year, despite the differences in the total number of cases in each circuit-year.  The Songer Dataset provides the appropriate weights for each circuit \citep{Songer2007}.}  This provides approximately 25,000 observations of cases in a random sample.   

The data for the individual judge level variables will come from the Federal Judicial Center biographical database with supplemental information coming from the U.S. Court of Appeals Judge Attribute Database \citep{FJC}. Measures of case salience are already coded into the \citeauthor{Songer2007} Dataset which reduces any additional coding, using the method developed by \cite{Hettinger2003}.  As noted above, case salience is likely to be an important factor in determining panel assignment.

\section{Methodology}\label{Methods}
For any given year $t$ there are 13 circuits $c$ of which there are a certain number of active circuit judges $J$, which is $J_{ct}$.  In theory, cases are assigned randomly to randomly assigned panels.  This theoretically creates two stages of randomization in the assignment of cases.  This theory is the primary premise that we are testing against.  The number of cases assigned to be decided in each circuit year are $N_{ct}$.  These cases are assigned in groups to panels, approximately once per month.  These panels will sit for approximately a week and hear the cases assigned for that month, and then convene in different panels the next month.  The Songer Database samples cases in a stratified way \citep{Songer2007}.  The sample $M$ for each circuit-year is $M_{ct}$ \citep{Songer2007}.  There is a split number of how these cases are sampled from 1961-2002, $M_{ct}=30$ and from 1925-1960, $M_{ct}=15$.  

Deriving the actual sampling distribution of those judges who appear in the Songer Dataset, is mathematically difficult, as well as being highly susceptible to omitted variable bias.\footnote{This omitted variable bias comes from no knowledge of judge's vacation schedule, illnesses, or reasons for recusals.  While this is not directly correlated with the outcome, it can cause a deviation in the distribution of the Songer Dataset, which will cause an error when comparing the Songer Dataset to the simulation.}  A much simpler way to accomplish this is the simulate the distribution of judges and use these empirical probabilities as a baseline for comparison to the Songer dataset \citep{Songer2007}.  By weighting these with the caseload statistics available, we can establish a baseline probability for judges and panels appearing in the simulated data. The simulation will establish a baseline of panel and case assignment probability which will show how under or over represented a judge is on the panels. 

We will build a circuit-judge-year dataset using data collected from the Federal Judicial Center which has biographical and career information for all federal judges \citep{FJC}.  This will create a simulated universe of judges who are available to be chosen for a particular panel.  This will be a dataset of judges  with one observation for each judge, per circuit, per term for the active judges on the circuit for a total of $\sum_{c=1}^{C}\sum_{t=1925}^{2002}J_{ct}$ judge-terms.  We will then create a similar simulated universe of ``cases.''  We will have one simulated case for each real case for a total of $\sum_{c=1}^{C}\sum_{t=1925}^{2002}N_{ct}$ cases.  For each circuit year, we will draw $Q_{ct}$ three-judge panels from the $J_{ct}$ available to be selected on a given circuit.  We will then draw a selection of cases, $S$, without replacement from the $N_{ct}$ cases in that circuit-term.  Each selection of cases will then be assigned randomly to a panel.  We will then sample from the simulated universe of cases in the same manner as Songer, without replacement, which is $M_{ct}$.  This gives us a sample of cases with three judges attached to it.  This mimics the procedure used in the real observations by \citet{Songer2007}.  We will then calculate $A_{jt}$, which is the number of times that judge $j$ appears on the $M$ cases in circuit $c$ during term $t$.  For example, if Judge Richard Posner of the Seventh Circuit appears on two of the cases in the observed data of the Songer sample, for the Seventh Circuit in 2000, he would have an $A_{jt}=2$.  From this we can calculate the empirical estimate of the fraction of the time each judge appears on a case decided by a circuit, which is\footnote{We don't multiply the denominator by three because a judge can only occupy one of the three seats on a panel. Note as well that $\Pi_{jt}$ will -- in theory -- be the same for every judge in a given circuit-year}:
\begin{align}\label{eq-probs}
\Pi_{jt}=\frac{A_{jt}}{M_{jt}}
\end{align}

We will then repeat the procedure above, approximately 100,000 times, which will create a repeated sample of this universe of cases and judges with the result being a 100,000 simulations of Songer's database.  This will give us an empirical estimate of how often each judge should appear in the Songer database, as well as an estimate of the variability of that estimate.  We can then use the mean of this estimate to create a set of estimates for $\bar{\Pi}_{jt}$ for all the judges in the data.\footnote{We use the mean to account for the stochastic variance produced from repeated sampling.}  We will then compare this to the actual frequency of times that each eligible judge appeared in the Songer database $P_{jt}$.  This will be calculated the same as in Equation \ref{eq-probs}.  This gives us the quantity of $P_{jt}-\bar{\Pi}_{jt}$ which is an empirical measure of how over- or under-represented each judge was on panels in each term she sat on the COA.  This will be the dependent variable for the subsequent testing of our hypotheses.  We will then apply a hierarchical generalized linear model with various judge level and circuit level effects.  This HLGM will be used to test our several hypotheses.  The HLGM will be used to find which independent variables can successfully predict the differences in the observed and simulated representation of the judges on the panels.  For example, we expect to find that the ideological distance between the judge and the Chief Judge will reduce the variance in the representation in the observed data. 

\section{Going Forward}\label{Further Research}
This project can provide several useful avenues of future research as well as provide important evidence to those in the policy community.  If panels are not randomly assigned, as we hypothesize, then the most important consequence to this is the potential loss of trust in the judicial system by the public.  While it is typically accepted in the political science literature that judges are ideological in the decisions that they make, one of the key constraints on this is the random assignment of COA judges to panels and cases.  If this constraint turns out to be a false one, this would signify that ideology is much more prevalent in the appellate process than previously thought.  This would also signify that one or more actors can influence the outcome of a COA case.  This is especially important due to the large numbers of cases which are terminated at the COA level.  In the year 2002, the Courts of Appeal heard 5,488 cases while the Supreme Court only heard 84 cases, slightly over 1.5\% of the number of cases that the COA heard \citep{judyearend,Songer2007}.   

Another important ramification is in the field of judicial politics in political science.  In recent years there has been an increase in assessing causality, and many recent articles have leveraged randomized panel and case assignment in the COA to learn about judicial decision making and racial and gender characteristics of COA judges \citep{Kastellec2010,Glynn2015,Farhang2014}.  If this research shows that panels and cases are not in fact randomly assigned, then those studies will have lost their leverage calling in to question their ability to assign causality.

%\newpage
\bibliographystyle{apsr}
\bibliography{panelassbib}
\end{document}
