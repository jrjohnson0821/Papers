\documentclass[12pt]{article}
\usepackage{setspace}
\usepackage{mathtools}
\usepackage{nicefrac}
\usepackage{fullpage}
\usepackage{times}
\usepackage{mathptmx}
\usepackage{graphicx}
\usepackage{natbib}
\usepackage{listings}
\usepackage{float}
\usepackage{wrapfig}
\usepackage{lscape}
\usepackage{hyperref}
\usepackage[affil-it]{authblk}
\usepackage{color}

%setup hyperlinks color
\definecolor{PSU}{RGB}{0,0,153}
\hypersetup{
	colorlinks=true,       % false: boxed links; true: colored links
	linkcolor=black,          % color of internal links (change box color with linkbordercolor)
	citecolor=PSU,        % color of links to bibliography
	filecolor=PSU,      % color of file links
	urlcolor=PSU           % color of external links
}

\title{Reassessing Random Assignment on the U.S. Courts of Appeals}

\author{\href{http://www.jeremyrjohnson.org/}{Jeremy R.\ Johnson}\\ \href{mailto:Jeremy.Johnson@psu.edu}{Jeremy.Johnson@psu.edu}}
\affil{Pennsylvania State University}
\date{\today}




\begin{document}
\maketitle
\thispagestyle{empty}

\doublespacing
\setcounter{page}{1}
%\pagenumbering{gobble}  %This turns off page numbers  

\begin{abstract}
	\medskip
	This is a test of the emergency abstract placeholder system.  If this had been an actual abstract, it would have briefly summarized the main points of the paper, so that a casual reader could quickly understand what the paper is about.  A high-quality abstract may also entice that person to read the entire paper, and possibly even to think about its contents.  This has been a test of the emergency abstract placeholder system. \\
\end{abstract}


\bibliographystyle{apsr}
\bibliography{panelassbib}

\end{document}
