\documentclass[JohnsonMADraft2.tex]{subfiles}
\begin{document}
\doublespacing

\subsection{Facial Validity}
As shown with the evidence above, this measurement model uses indicators that have been shown through previous literature to indicate judicial independence. In addition, these are all indicators that are institutional in nature, rather than measuring some concept that would measure both \textit{de facto} and \textit{de jure} independence. Rather than measuring the amount of docket control that a court exercises, I examine the control that is constitutionally or statutorily authorized to the court.


\todo{Clearly Define This, \citet{Jackman2008}}
%Ginsburg and Melton Replication
\subsection{Convergent Validity -- Ginsburg and Melton Replication}
As a test of convergent validity, I have replicated the additive measure discussed above from \citet{Melton2014}.  \citet{Melton2014} gathered data from the Comparative Constitutions Project (CCP) to code indicators of \textit{de jure} judicial independence using the text of national constitutions.  I replicated their project and applied it to the American states by using the texts of state constitutions gathered from the NBER/Maryland State Constitutions Project (SCP) \citep{Wallisnber}.\footnote{The SCP did not have text for all current state constitutions, for the states that were unavailable, the official texts were downloaded from the state government website \citep{Wallisnber}.}  Of the many variables available in the CCP, \citet{Melton2014} used five questions to code an additive index of \textit{de jure} judicial independence.  The five questions consisted of: a statement of judicial independence, lifetime terms, selection procedures, removal procedures, removal conditions, and salary insulation.  The questions used by CCP are available in Appendix \ref{CCPCode}.  This data is coded using exclusively the text of the constitutions in question, as such, my replication has followed the same procedures.

	
\end{document}