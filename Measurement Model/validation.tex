\documentclass[JohnsonMADraft3.tex]{subfiles}
\begin{document}
\doublespacing

\subsection{Facial Validity}
As shown with the evidence above, this measurement model uses indicators that have been shown through previous literature to indicate judicial independence. In addition, these are all indicators that are institutional in nature, rather than measuring some concept that would measure both \textit{de facto} and \textit{de jure} independence. Rather than measuring the amount of docket control that a court exercises, I examine the control that is constitutionally or statutorily authorized to the court.

%Ginsburg and Melton Replication
\subsection{Convergent Validity -- Ginsburg and Melton Replication}
As a test of convergent validity, I have replicated the additive measure discussed above from \citet{Melton2014}.  \citet{Melton2014} gathered data from the Comparative Constitutions Project (CCP) to code indicators of \textit{de jure} judicial independence using the text of national constitutions.  I replicated their project and applied it to the American states by using the texts of state constitutions gathered from the NBER/Maryland State Constitutions Project (SCP) \citep{Wallisnber}.\footnote{The SCP did not have text for all current state constitutions, for the states that were unavailable, the official texts were downloaded from the state government website \citep{Wallisnber}.}  Of the many variables available in the CCP, \citet{Melton2014} used five questions to code an additive index of \textit{de jure} judicial independence.  The five questions consisted of: a statement of judicial independence, lifetime terms, selection procedures, removal procedures, removal conditions, and salary insulation.  The questions used by CCP are available in Appendix \ref{CCPCode}.  This data is coded using exclusively the text of the constitutions in question, as such, my replication has followed the same procedures.

% latex table generated in R 3.1.2 by xtable 1.7-4 package
% Fri Apr 17 11:58:40 2015
\begin{table}[ht]
	\centering\caption{Correlation of Measures}\label{Correlation}
	\begin{tabular}{rccc}
		\hline
		& Additive & IRT & \citeauthor{Melton2014} \\ 
		\hline
		Additive & -- & $-.95^{***}$ &$-0.45^{**}$ \\ 
		IRT & $-0.95^{***}$ & -- & $0.52^{***}$ \\ 
		\citeauthor{Melton2014} & $-0.45^{**}$  &  $0.52^{***}$ & --\\ 
		\hline
		\textit{Note:}  & \multicolumn{3}{l}{$^{*}$p$<$0.1; $^{**}$p$<$0.05; $^{***}$p$<$0.01} \\
	\end{tabular}
\end{table}

The index created by \citet{Melton2014} was correlated with both the five indicator model and the additive index with my scaling using Polychloric Correlation \citep{Olsson1979}.\footnote{Correlation was also calculated using Pearson's $R$ and Kendall's Tau $B$ and were not significantly different.}  The data use for these calculations are located in Appendix \ref{GMData}.  Table \ref{Correlation} shows these correlations.  All are statistically significant, however, they are not overwhelmingly 

\subsection*{Discriminant Validity}
To test for any other factors that could potentially be associated with the indicators used for this measure of judicial independence I have correlated both the IRT measure and the additive measure with \citet{Enns2013}'s measure of mood.  The results are shown in Table \ref{EKCor}.  Divergent validity is contrasted with convergent validity in that it ensures that the measure in question does not measure things that it is not supposed to measure \citep{Campbell1959,Jackman2008}.  As shown in Table \ref{EKCor} this criteria is satisfied to a respectable degree.

% latex table generated in R 3.1.2 by xtable 1.7-4 package
% Fri Apr 17 20:25:11 2015
\begin{table}[ht]
	\centering\caption{Correlation of Estimates with \citet{Enns2013}}\label{EKCor}
	\begin{tabular}{rcccc}
		\hline
		 &  Mood &  2 Dimension Mood & Posterior Estimates & Additive \\ 
		\hline
		 Mood & -- & $0.58^{***}$ & $0.40^{**}$ & $-0.40^{**}$ \\ 
		 2 Dimension Mood &  $0.58^{***}$ & -- & $0.16$ & $-0.15$ \\ 
		Posterior Estimates &  $0.40^{**}$  &  $0.16$  & -- & $-0.95^{***}$ \\ 
		Additive & $-0.40^{**}$  & $-0.15$  & $-0.95^{***}$ & -- \\ 
		\hline
		\textit{Note:}  & \multicolumn{4}{l}{$^{*}$p$<$0.1; $^{**}$p$<$0.05; $^{***}$p$<$0.01} \\
	\end{tabular}
\end{table}
\end{document}