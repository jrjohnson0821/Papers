\documentclass[Johnson MA Draft 2.tex]{subfiles}
\begin{document}
% A Paragraph
For more than two centuries, judicial institutions have played a key role in the operation of democratic systems of government.  An independent court ``counteracts the logic of `winner-takes-all' where whoever wins the election wins everything. Thanks to the mechanism of constitutional adjudication, the electoral victory is not an `all or nothing' game''\citep[1685]{Ferejohn2003}.  How does this change when the court is selected by the majority as well?  When those courts are elected alongside the representatives on partisan ballots, do they have the same incentives for majoritarian rule as governors and state legislators?

%B Paragraph
The ability of judicial institutions to be effective hinges on their ability to rule against the government, as well as make legal decisions free of influence from other actors such as corporations or interest groups.  \citet{Ferejohn2003} characterizes this as providing a voice against what John Adams referred to as the ``tyranny of the majority'' \citep{Adams1794}.  The cost to this independence is that the government, and citizens at large have a reduced set of options in holding judges accountable for their decisions.  If a Congressman or State Legislator makes decisions or casts votes in opposition to the publics interest they can simply vote them out of office.  In an independent judiciary this option is generally removed with the exception of impeachment for criminal acts.

% C Paragraph
Judicial independence is often discussed as a cause for of continuity of the rule of law, or as   Judicial independence has also been used in many studies of comparative politics as a key indicator of regime stability, economic growth, and most prominently protection of human rights \citep[1]{Linzer2014}.  Many scholars agree that an independent judiciary is necessary for the protection of both human rights and political rights \citep{Keith2002a,Keith2002b,Howard2004,Russell2001}.\footnote{For a more extensive review of judicial independence in the human rights literature see \citep[Footnote 1]{Keith2002b}.}  When examining human rights, judicial independence has often been an important concept; some form of judicial independence is incorporated in the U.S. State Department's Human Rights Scores as well as multiple variables in the POLITY IV dataset \citep{Cingranelli2008, Polity,Howard2004}.To take but one example, \citet{Keith2002b}'s influential study clearly demonstrates a strong relationship between judicial independence and human rights protections and the World Economic Forum's \textit{Global Competitiveness Report} uses judicial independence as a key indicator in the first pillar of their study \citep{WEFGLR2014}.\footnote{\textit{The Global Competitiveness Report} is a perception based survey of business experts for 144 countries.  Judicial independence is covered by the question ``In your country, to what extent is the judiciary independent from influences of members of government, citizens, or firms?''}  The World Economic Forum uses increased judicial independence as an indicator of a better environment for companies to do business.  When a foreign company is in a dispute with a local government, an independent judiciary increases the chances that the company will be treated fairly in any legal proceeding rather than the court simply rubber-stamping the government's decision.  

Judicial independence can be a promoter of other societal benefits such as foreign investment and subsequent economic growth, and consolidation of democracy \citep[9]{Rios2006}.  \citet{Hayo2007} discuss some of the important factors relating to \textit{de jure} judicial independence, which they refer to as formal independence.  First on that list is credibility of the government.  Governments must make credible commitments to an independent judiciary in order to assure its citizens that they will respect the constitutional rights that they are guaranteed.  Formal recognition of an independent judiciary is an important part of that credible commitment.   Credibility is a large component of the perceptional surveys that are conducted in research on corruption and human rights.

%D Paragraph
As \citet{Tiede2006} notes, judicial independence is often cited, but rarely understood.  Much of the previous research into judicial independence lacks consistency and coherence.  Many scholars have used different indicators of judicial independence based on their specific research question.  For \textit{de facto} judicial independence, this is discussed at length in \cite{Rios2014}.  As a result of this, I propose a new theory which combines those indicators which have shown to be informative to \textit{de jure} judicial independence in the American states, based principally on selection and retention methods.

%E Paragraph
This paper presents a new theory of \textit{de jure} judicial independence in the American states.  This paper also presents a new measure of modeling \textit{de jure} judicial independence in the American states.  The paper proceeds as follows: Section \ref{Theory} presents a new theory of judicial independence, Section \ref{Indicators} presents the indicators of this theory, Section \ref{Methods} the corresponding measurement model, Section \ref{Validation} tests the validity of this measure against other measures, and Section \ref{Application} concludes with proposed applications of this measure to other research.	
\end{document}

