\documentclass[12pt]{article}
\usepackage{setspace}
\usepackage{mathtools}
\usepackage{nicefrac}
\usepackage{fullpage}
\usepackage{times}
\usepackage{mathptmx}
\usepackage{graphicx}
\usepackage{natbib}
\usepackage{listings}
\usepackage{float}
\usepackage{wrapfig}
\usepackage{lscape}
\usepackage{hyperref}
\usepackage[affil-it]{authblk}

\title{Draft of Methods Section}

\author{\href{http://www.jeremyrjohnson.org/}{Jeremy R.\ Johnson}\\ \href{mailto:Jeremy.Johnson@psu.edu}{Jeremy.Johnson@psu.edu}}
\affil{Pennsylvania State University}
\date{\today}



\begin{document}
\maketitle
\thispagestyle{empty}

\doublespacing
\setcounter{page}{1}
%\pagenumbering{gobble}  %This turns off page numbers  
\subsection*{Model}
I assume that the observed indicators for each state-year are functions of a unidimensional latent variable that represents the level of \textit{de jure} judicial independence.  For each state-year observation, let $i$ index the state and $t$ index the year.  or each model, there are $J$ indicators $J=1$,...,$J$ each of which is ordinal. My goal is to estimate each $\theta_{it}$, which is the latent level of \textit{de jure} judicial independence of each state $i$ in year $t$.

Let $i=1$,...,$N$ index cross-sectional units and $t=1$,...,$T$ index time periods.  In each time, period, I observe values $y_{ij}$ for each of $j=1$,...,$J$ indicators for each unit.  Each indicator is ordinal in nature and can take on $K_j$ values.  The responses to each of the items depend on a single latent variable $\theta_{it}$, which may vary across units and over time. I assume that all indicators are independently draw from a logistic distribution.  
\end{document}
