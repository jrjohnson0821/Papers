\documentclass[JohnsonMADraft2.tex]{subfiles}
\begin{document}
\subsection*{Data}
The model uses institutional data on the courts of last resort used by the 50 states in the United States.  There are 52 courts of last resort in the United States.\footnote{The court of last resort for the District of Columbia is the District of Columbia Court of Appeals.  As the District of Columbia is not a state and its courts fall under the federal system, it is omitted in this study.  The D.C. Superior Court is the trial court and refers all appeals to the D.C. Court of Appeals.  Its judges are appointed by the President and confirmed by the Senate to fifteen year terms.}  Each state has a court while Texas and Oklahoma have specialized courts for civil and criminal.  In this study these court systems will be examined together as they have identical institutional arrangements.  

I create three models of \textit{de jure} judicial independence.  The first uses the first four indicators discussed below.  This model comprises the 50 states from 1800-2012.  The second model uses five indicators, with the addition of a docket control indicators from 1800-2012.  The reason for the two different models is the lack of available data on docket control prior to 1970.  The states had a variety of control over their dockets from 1970 through the present.  However, between 1940 and 1970 the data is nearly non-existent due to a lack of reliable surveys of state court organizations.  The post-1970 data was provided by the Bureau of Justice Statistics (BJS) and the National Center for State Courts (NCSC) \citep{BJS1993,BJS1998,BJS2004}.  

The third model specification is a regime change model, which only includes only state-year observations that indicate a change in any of the four indicators used in the primary specification.  

Prior to 1940, in addition to the data being non-existent, there was also much dispersion in the interpretation of docket control based on the individual states.  In many of these states, even if there was an ``appeal by right,'' some states interpreted this as mandatory jurisdiction, while others did not.  These discrepancies lead toward a more \textit{de facto} indicator than is being addressed in this paper.  With the standardized definitions used in the BJS and NCSC reports, it becomes useful to add this indicator in for a post-1970 model, but as shown below, the wider variance in selection/retention methods and term lengths, still make a study of a longer time span useful, despite the lack of jurisdictional control data.  

\subsection*{Indicators}
All of the indicators are ordinal variables with 0 representing the most independent moving towards least independent.  \texttt{APPT} is coded from 0-4, which is the appointment method that is used for a judge's initial appointment.  \texttt{TERM1} is coded from 0-3 and represents the length of the judge's initial term.  \texttt{TERM2} is coded in a similar manner as \texttt{TERM1} and represents the length of the judge's subsequent terms after their first retention or reappointment. 
%-------------------------------------------------------------
\begin{table}[!htb]\singlespacing\centering
	\caption{List of Indicators}\label{Indicators}
	
	\begin{tabular}{ccl}\hline
		\textbf{Variable}	&		&	\textbf{Code}	\\\hline\hline
		APPT	&		&	Appointment Method	\\
		0	&	-	&	Gubernatorial Appointment	\\
		1	&	-	&	Commission System	\\
		2	&	-	&	Legislative Appointment	\\
		3	&	-	&	Non-Partisan Election	\\
		4	&	-	&	Partisan Election	\\\hline
		TERM1	&		&	Initial Term Length	\\
		0	&	-	&	Lifetime or Quasi-Lifetime Term\\
		1	&	-	&	$\geq10$ Years	\\
		2	&	-	&	7-9 Years	\\
		3	&	-	&	4-6 Years	\\
		4	&	-	&	1-3 Years	\\\hline
		TERM2	&		&	Subsequent Term Length	\\
		0	&	-	&	Lifetime Term	\\
		1   &   -   &   $\geq10$ Years \\
		2	&	-	&	7-9 Years	\\
		3	&	-	&	4-6 Years	\\
		4	&	-	&	1-3 Years	\\\hline
		RETELE	&		&	Method of Retention	\\
		0	&	-	&	No Retention Election	\\
		1	&	-	&	Retention Election	\\
		2	&	-	&	Contested Retention Election	\\\hline
		DOCKET	&		&	Docket Control Discretion	\\
		0	&	-	&	Complete Discretion	\\
		1	&	-	&	Mixed Discretion	\\
		2	&	-	&	Mandatory Docket	\\\hline
	\end{tabular}
	
\end{table}
%--------------------------------------------------------------
There is one major difference however, in that 0 represents those states in which the judge is initially appointed to a lifetime term\footnote{The only major time this is an issue is in New Jersey, in which judges are initially appointed to a seven year term and then reappointed to a lifetime term.  However, since New Jersey is represented with a seven year term initially, this is different than if they were initially appointed to a lifetime term and therefore their subsequent term is greater than ten years.}.  In many states, these two terms are identical.  These differences are primarily in merit selection states which have one to three year terms followed by a retention election and then have longer terms after the first retention election.  Coding term lengths as ordinal rather than continuous, allows for consistent interpretation across the model.  \texttt{RETELE} is coded from 0-2 and indicates the method of retention.  0 indicates a simple reappointment procedure with no election.  1 represents an uncontested retention election, most common under the merit selection system, but also used in states that utilize partisan or non-partisan election systems.  2 indicates a contested retention election, either partisan or non-partisan.  \texttt{DOCKET} represents the amount of discretion that a court has over its docket.  0 is a completely discretionary docket, while 2 is no discretion.  A 1 indicates a mix of discretionary and mandatory appeals. Table \ref{Indicators} shows the coding for all levels of each indicator. 
\end{document}