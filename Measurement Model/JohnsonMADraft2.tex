\documentclass[12pt]{article}
\usepackage{setspace}
\usepackage{mathtools}
\usepackage{nicefrac}
\usepackage{fullpage}
\usepackage{times}
\usepackage{mathptmx}
\usepackage{amsmath}
\usepackage{graphicx}
\usepackage{natbib}
\usepackage{listings}
\usepackage{float}
\usepackage{wrapfig}
\usepackage{booktabs}
\usepackage{lscape}
\usepackage{hyperref}
\usepackage{geometry}
\usepackage{longtable}
\usepackage{pdflscape}
\usepackage[affil-it]{authblk}
\usepackage{todonotes}
%\usepackage[latin1]{inputenc}
\usepackage{tikz}
\usetikzlibrary{shapes,arrows,positioning}
\usepackage{amsmath}
\usetikzlibrary{arrows}
\usepackage{caption, subcaption}
\usepackage{subfiles}
%setup hyperlinks color
\usepackage{color}
\definecolor{PSU}{RGB}{0,0,153}
\hypersetup{
	colorlinks=true,       % false: boxed links; true: colored links
	linkcolor=black,          % color of internal links (change box color with linkbordercolor)
	citecolor=PSU,        % color of links to bibliography
	filecolor=PSU,      % color of file links
	urlcolor=PSU           % color of external links
}


\title{Measuring Judicial Independence in the American States: A Latent Variable Approach}

\author{\href{mailto:Jeremy.Johnson@psu.edu}{Jeremy R.\ Johnson}\thanks{All data and code required for replication is available on GitHub at: \url{https://github.com/jrjohnson0821/stateslatent}}}
\affil{Pennsylvania State University}
\date{\today}

\begin{document}
\maketitle
\thispagestyle{empty}
	
\begin{abstract}
Judicial independence and judicial accountability are commonly understood to exist in tension with one another. Many scholars and professionals, among them the American Bar Association, believe that the judiciary should be entirely independent from any outside influence, be it electoral, executive, or legislative. A contrary view is that judges have become too independent, and need to be “reigned in” by those who can affect control over them, primarily through judicial elections. The core of this disagreement lies with differing understandings of what constitutes “independence” and “accountability.” Absent clear conceptualization and measurement of these concepts, much of the normative debate over judicial independence reduces to disagreement over terms. Despite its manifest importance, there is not currently a comprehensive measure of \textit{de jure} judicial independence in American or comparative politics. I develop and use a latent-variable model to score states' \textit{de jure} judicial independence. This measure will be useful in future studies of judicial independence which focus on \textit{de facto} judicial independence.
\end{abstract}
	
	
\pagebreak\doublespacing
\setcounter{page}{1}

\section{Introduction}\label{Intro}
% A Paragraph
For more than two centuries, judicial institutions have played a key role in the operation of democratic systems of government.  An independent court ``counteracts the logic of `winner-takes-all' where whoever wins the election wins everything. Thanks to the mechanism of constitutional adjudication, the electoral victory is not an `all or nothing' game'' \citep[1685]{Ferejohn2003}. Normatively, courts provide a secondary outlet for minorities to find resolution of legal problems they cannot otherwise achieve through majority representation.  How does this change when the court is selected by the majority as well?  When those courts are elected alongside the representatives on partisan ballots, do they have the same incentives for majoritarian rule as governors and state legislators?

%B Paragraph
The ability of judicial institutions to be effective hinges on their ability to rule against the government, as well as make legal decisions free of influence from other actors such as corporations or interest groups.  \citet{Ferejohn2003} characterizes this as providing a voice against what John Adams referred to as the ``tyranny of the majority'' \citep{Adams1794}.  The cost to this independence is that the government, and citizens at large have a reduced set of options in holding judges accountable for their decisions.  If a Congressman or State Legislator makes decisions or casts votes in opposition to the publics interest they can simply vote them out of office.  In an independent judiciary this option is generally removed with the exception of impeachment for criminal acts.

% C Paragraph
Judicial independence is often discussed as a cause for of continuity of the rule of law, judicial independence has also been used in many studies of comparative politics as a key indicator of regime stability, economic growth, and most prominently protection of human rights \citep[1]{Linzer2014}.  Many scholars agree that an independent judiciary is necessary for the protection of both human rights and political rights \citep{Keith2002a,Keith2002b,Howard2004,Russell2001}.\footnote{For a more extensive review of judicial independence in the human rights literature see \citep[Footnote 1]{Keith2002b}.}  When examining human rights, judicial independence has often been an important concept; some form of judicial independence is incorporated in the U.S. State Department's Human Rights Scores as well as multiple variables in the POLITY IV dataset \citep{Cingranelli2008, Polity,Howard2004}.To take but one example, \citet{Keith2002b}'s influential study clearly demonstrates a strong relationship between judicial independence and human rights protections.  The World Economic Forum's \textit{Global Competitiveness Report} uses judicial independence as a key indicator in the first pillar of their study of the global business climate \citep{WEFGLR2014}.\footnote{\textit{The Global Competitiveness Report} is a perception based survey of business experts for 144 countries.  Judicial independence is covered by the question ``In your country, to what extent is the judiciary independent from influences of members of government, citizens, or firms?''}  The World Economic Forum uses increased judicial independence as an indicator of a better environment for companies to do business.  When a foreign company is in a dispute with a local government, an independent judiciary increases the chances that the company will be treated fairly in any legal proceeding rather than the court simply rubber-stamping the government's decision.  Judicial independence can be a promoter of other societal benefits such as foreign investment and subsequent economic growth, and consolidation of democracy \citep[9]{Rios2006}.  

\citet{Hayo2007} discuss some of the important factors relating to \textit{de jure} judicial independence, which they refer to as formal independence.  First on that list is credibility of the government.  Governments must make credible commitments to an independent judiciary in order to assure its citizens that they will respect the constitutional rights that they are guaranteed.  Formal recognition of an independent judiciary is an important part of that credible commitment.   Credibility is a large component of the perceptional surveys that are conducted in research on corruption and human rights.

%D Paragraph
``Judicial independence is often cited, but rarely understood,'' \citep[1]{Tiede2006}.  Much of the previous research into judicial independence lacks consistency and coherence in both methods and normative definitions.  Many scholars have used different indicators of judicial independence based on their specific research question.  As \citeauthor{Rios2014} point out, this lack of clarity of measurement makes it difficult to realistically interpret influence that formal rules have on judicial independence \citep[2]{Rios2014}.  For \textit{de facto} judicial independence, this is discussed at length in \cite{Rios2014}.  As a result of this, I propose a new theory which combines those indicators which have shown to be informative to \textit{de jure} judicial independence in the American states, based principally on selection and retention methods.

%E Paragraph
This paper presents a new theory of \textit{de jure} judicial independence in the American states.  This paper also presents a new measure of modeling \textit{de jure} judicial independence in the American states.  The paper proceeds as follows: Section \ref{Theory} presents a new theory of judicial independence, Section \ref{Indicators} presents the indicators of this theory, Section \ref{Methods} the corresponding measurement model, Section \ref{Validation} tests the validity of this measure against other measures, and Section \ref{Application} concludes with proposed applications of this measure to other research.	

\section{A Theory of \textit{De Jure} Judicial Independence}\label{Theory}
\subfile{theory.tex}

\section{Indicators}\label{Indicators}
\subfile{indicators.tex}

\section{Methodology}\label{Methods}
\subfile{data.tex}

\section{Model}
\subfile{model.tex}

<<<<<<< HEAD
\section{Results}\label{Results}
=======
\section{Results}
>>>>>>> origin/master
\subfile{results.tex}

\section{Validation}\label{Validation}
\subfile{validation.tex}

\section{Application}\label{Application}
\subfile{application.tex}

\singlespacing
\bibliographystyle{apsr}
\bibliography{measurementbib}
\appendix
\section{Coding Notes}\label{CodingNotes}
\begin{itemize}
	\item Arkansas- In 1868, the Chief Justice was appointed by the Governor, with Senate Consent, with the Associate Justices elected by the people.  This is coded as partisan election since the majority of the justices are elected.
	
	\item Connecticut- from 1784 through 1818 Judges were appointed by the Legislature, but the tenure in office was dependent upon the action of the legislature.  These years are coded as missing and subsequently dropped from the dataset.
	
	\item Delaware- The original court of last resort was the Court of Appeals, and the data is coded as such.  The Supreme Court was created by constitutional amendment in 1951, at which point the data is coded to reflect the Supreme Court.
	
	\item Michigan- In 1939, a constitutional amendment passed calling for non-partisan elections for judges, except the Supreme Court, which would continue to be nominated at party conventions.  This is coded as a non-partisan following practice in this field.
	
	\item New Jersey- AJS Data does not reflect the Constitutional amendment in 1983, which extended subsequent terms to a term of good behavior.
	
	\item New Mexico- Judges are elected for the remainder of the unexpired term.  This could be up to 8 years, so it was coded as such.
	
	\item Oklahoma is examined individually, but both courts have mandatory jurisdiction.-2004
	
	\item Pennsylvania- From 1874 to 1968 Supreme Court justices were elected to twenty-one year terms and were not eligible for reelection.  This is coded as a 0 for the subsequent term.
	
	\item Tennessee- Constitution says qualified electors, however, by executive order the governor has created a nominating commission
	
	\item Texas is examined individually, SC has discretionary jurisdiction, but COA has mandatory in sentencing issues, so this is coded as mixed.- 2004
	
	\item Retention Elections- Gubernatorial reappointment, judicial commission reappointment, legislative reappointment are coded as No Retention Elections.
	
	\item Docket Control- This is established by looking at Criminal and Civil cases.  Administrative agency decisions are omitted, as well as death penalty cases.
	
	\item Docket Control- Observations between BJS reports are assumed to be static.
\end{itemize}	

\section{Comparative Constitutions Project Variables}\label{CCPCode}
\begin{itemize}
	\item Statement of Judicial Independence -- \textbf{[JUDIND]} -- Does the constitution contain an explicit declaration regarding the independence of the central judicial organ(s)? 
	\item Life Term -- \textbf{[SUPTERM]} -- What is the maximum term length for judges for the highest ordinary court? 
	\item Selection Procedure -- \textbf{[SUPNOM,SUPAP]} -- Who is involved in the nomination of judges to the highest ordinary court? and Who is involved in the approval of nominations to the highest ordinary court? 
	\item Removal Procedures -- \textbf{[JREM, JREMPRO, JREMFIRP, JREMSECP, JREMBOTP, JREMAP]} -- Are there provisions for dismissing judges? and Who can propose the dismissal of judges? and Who can approve the dismissal of judges?
	\item Removal Conditions -- \textbf{[JREMCON]} -- Under what conditions can judges be dismissed?
	\item Salary Insulation -- \textbf{JUDSAL} -- Does the constitution explicitly state that judicial salaries are protected from governmental intervention?
\end{itemize}
\end{document}
