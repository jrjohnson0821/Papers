\documentclass[12pt]{article}
\usepackage{setspace}
\usepackage{mathtools}
\usepackage{nicefrac}
\usepackage{fullpage}
\usepackage{times}
\usepackage{mathptmx}
\usepackage{graphicx}
\usepackage{natbib}
\usepackage{listings}
\usepackage{float}
\usepackage{wrapfig}
\usepackage{booktabs}
\usepackage{lscape}
\usepackage{hyperref}
\usepackage{geometry}
\usepackage{longtable}
\usepackage{pdflscape}
\usepackage[affil-it]{authblk}
\usepackage{todonotes}
\usepackage{caption, subcaption}
\usepackage{subfiles}
%setup hyperlinks color
\usepackage{color}
\definecolor{PSU}{RGB}{0,0,153}
\hypersetup{
	colorlinks=true,       % false: boxed links; true: colored links
	linkcolor=black,          % color of internal links (change box color with linkbordercolor)
	citecolor=PSU,        % color of links to bibliography
	filecolor=PSU,      % color of file links
	urlcolor=PSU           % color of external links
}


\title{Measuring Judicial Independence in the American States: A Latent Variable Approach}

\author{\href{mailto:Jeremy.Johnson@psu.edu}{Jeremy R.\ Johnson}\thanks{All data and code required for replication is available on GitHub at: \url{https://github.com/jrjohnson0821/stateslatent}}}
\affil{Pennsylvania State University}
\date{\today}

\begin{document}
\maketitle
\thispagestyle{empty}
	
\begin{abstract}
Judicial independence and judicial accountability are commonly understood to exist in tension with one another. Many scholars and professionals, among them the American Bar Association, believe that the judiciary should be entirely independent from any outside influence, be it electoral, executive, or legislative. A contrary view is that judges have become too independent, and need to be “reigned in” by those who can affect control over them, primarily through judicial elections. The core of this disagreement lies with differing understandings of what constitutes “independence” and “accountability.” Absent clear conceptualization and measurement of these concepts, much of the normative debate over judicial independence reduces to disagreement over terms. Scholars of judicial politics commonly recognize two types of judicial independence, \textit{de jure} which describes the institutional arrangements and structures of courts and constitutions, and \textit{de facto} which describes what actually happens in those institutions in practice.  Despite its manifest importance, there is not currently a comprehensive measure of de jure judicial independence in American or comparative politics. I develop and use a latent-variable model to score states' \textit{de jure} judicial independence. This measure will be useful in future studies of judicial independence which focus on \textit{de facto} judicial independence.
\end{abstract}
	
	
\pagebreak\doublespacing
\setcounter{page}{1}

\section{Introduction}\label{Intro}
\subfile{introduction.tex}

\section{A Theory of \textit{De Jure} Judicial Independence}\label{Theory}
\subfile{theory.tex}

\section{Indicators}\label{Indicators}
\subfile{indicators.tex}

\section{Methodology}\label{Methods}
\subfile{data.tex}

\section{Model}
\subfile{model.tex}

\newpage\section{Validation}\label{Validation}
\subfile{validation.tex}

\section{Application}\label{Application}
\subfile{application.tex}

\singlespacing
\bibliographystyle{apsr}
\bibliography{measurementbib}
\appendix
\subfile{codingnotes.tex}

\subfile{gmrepappend.tex}
\end{document}
